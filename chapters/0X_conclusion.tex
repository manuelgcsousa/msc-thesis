\chapter{Conclusion} \label{conclusion}

With this document, the main objective was to conduct a study and analysis about what this problem involves and what could, in any way, help create an adequate solution. Furthermore, the first approach to the problem was documented, giving special attention to the case studies and a first sketch of what will become a new language for linguists.

Regarding to the approaches that were taken, it is quite clear that some parts are still at an early stage of development, and require more time - mainly the design of the students \textsc{DSL}, which is still very verbose in comparison to what was projected. The next step, which was already taken, is the creation of the meta-grammar that processes the information written by the teacher + student, and generates the \textsc{ANTLR} instructions based on the defined structure. The generated grammar will be based on the \textsc{DSL} structure that was used for the case studies. Afterwards, with a functional validator, the goal is to build a system with a user interface that allows to visualize the syntax tree of the input sentence, helping when it comes to analyse each segment individually.

\section{Working Plan}
The outlined work plan for this master thesis will consist of six phases. Each phase will include the conclusions of the previous phases and build upon the knowledge gained in each one.

\begin{description}
	\item[Phase 1] Bibliographic search in applying attribute grammars to linguistics, and study the tools that are already available.
	\item[Phase 2] Design the domain specific language (\textsc{DSL}) with all the requeriments previously mentioned.
	\item[Phase 3] Create the language translator in \textsc{ANTLR}.
	\item[Phase 4] Create the user interface.
	\item[Phase 5] Test and required adjustments.
	\item[Phase 6] Write Thesis.
\end{description}

For a better visualization, it was created a diagram based on all the different phases, divided into the months.

\begin{table}[!ht]
\centering
\caption{Activities Plan detailed}
\vspace{0.2cm}
\label{roadmap}
\scalebox{0.5}{
\def\S{\cellcolor{gray!75}}
\begin{tabular}{cp{0.5\linewidth}:cccc:cccc:cccc:cccc:cccc:}
\hline
\cline{1-14} % linha entre os anos e meses
Phase & Phase Description & Sep & Oct & Nov & Dec & Jan & Feb & Mar & Apr  & May & Jun & Jul & Aug \\
\hline
1& Bibliographic search & \S & \S & & &  &  &  &  &  &  &  & &   \\[4ex] % entre & & é o conteúdo da coluna
\\[0.5ex] % espaço em branco entre linhas
2& Designing the \textsc{DSL} & & & \S & \S & &  &  &  &  & & & \\[4ex]
\\[0.5ex]
3& Creating the language translator\ & & & & & \S & \S & \S & \\[4ex]
\\[0.5ex]
4& Creating the user interface. &  &  &  &  & & & & \S & \S &  &  & \\[4ex]
\\[0.5ex]
5& Testing and adjustments &  &  &  &  &  & & & & & \S & & & \\[4ex]
\\[0.5ex]
6& Finishing writing the thesis &  &  &  &  &  &  &  &  &  & & \S & \S & \\[4ex]
\end{tabular} }
\end{table}