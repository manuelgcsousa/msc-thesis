This document presents the topic “Applying Attribute Grammars to teach Linguistic Rules”, at Universidade do Minho in Braga, Portugal.
This thesis is focused on using the formalisms of attribute grammars in order to create a tool to help linguistic students learn the different rules of a natural language. 
The system developed, named \textbf{Lyntax}, consists in a processor for a domain specific language which intends to enable the user to specify different kinds of
sentence structures, and afterwards, test various phrases against said structures.
The processor validates and evaluates the input given, generating a grammar which is specific to a previously chosen sentence.
Lastly, using \textsc{ANTLR}, a parser is generated for that specific grammar referred above. The processor built by \textsc{ANTLR} also creates a syntax tree that is presented to the user for analysis purposes.

An interface that supports the specification of the language (written in Lyntax DSL) was built, also allowing the use of the processor and the generation of the specific grammar,
exempting the user from knowing the details of the process.

Within this document, the focus will be primarly dedicated to the analysis of the system and how each block was built.
Different examples of the processor in action will be shown and explained.\\[0.1cm]

%The main goal is to create a new DSL with a simple notation,
%suitable for any person that does not have any experience with common programming languages elements, that will allow to define linguistic exercices. 
%Furthermore, it was created a user interface that supports that same tool, granting a more visual experience for the user.\\[0.1cm]

\noindent \textbf{Keywords}: Linguistic, Natural Language Processing, Attribute Grammar