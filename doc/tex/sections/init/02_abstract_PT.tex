Este documento refere-se a uma dissertação sobre o tópico ``Aplicar Gramáticas de Atributos no ensino de Regras de Linguística'', 
e será concluída na Universidade do Minho em Braga, Portugal.
Esta dissertação pretende focar-se no uso dos formalismos das gramáticas de atributos de maneira a criar uma ferramenta que ajude os alunos de linguística a aprender 
as diversas regras da língua natural.

O sistema desenvolvido, denominado de \textbf{Lyntax}, consiste em um processor para uma linguagem de domínio específico cujo objetivo é o de permitir ao seu utilizador
a possibilidade de especificar diversas estruturas de frases, e posteriormente, testar frases contra essas mesmas estruturas.
O processador valida e avalia o input recebido, gerando uma gramática específica à frase previamente escolhida.
Por fim, usando uma ferramenta como o \textsc{ANTLR}, um parser é gerado para a gramática específica acima referida. O processador construído pelo \textsc{ANTLR} também gera a árvore de syntax que é apresentada ao utilizador com o intuito de ser analisada.

Foi também criada uma interface que suporta a especificação da linguagem, permitindo também o uso do processador e a geração da gramática específica,
abstraindo assim o utilizador de quaisquer tipo de cálculos.

Neste documento, o focus primário será dedicado à análise do sistema e como cada bloco foi construído.
Diferentes exemplos de uso do processador serão apresentados e explicados.\\[0.1cm]

%O principal objetivo é a criação de uma \textsc{DSL} com uma notação simples, 
%adequada a qualquer pessoa que não tenha experiência com elementos comuns das linguagens de programação,
%que permita a definição de exercícios de linguística. 
%Para além disso, é esperada a criação de uma interface que sirva de suporte a essa mesma ferramenta, permitindo assim uma experiência visual ao utilizador.\\[0.1cm]
    
\noindent \textbf{Palavras-chave}: Linguística, Processamento de Língua Natural, Gramáticas de Atributo