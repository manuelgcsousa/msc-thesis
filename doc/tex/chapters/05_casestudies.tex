\chapter{Case Studies} \label{case_study}
In order to better explain the solution architecture presented in this thesis, some concrete examples will be presented in this section. 
The main idea is to show the specifications used by the teacher and by the student and how the generator processor verifies the correcteness of the student sentences. 
The DSL design consists in slicing the sentence into parts, and each part can have subparts.

\textbf{Lyntax}, the word that combines the terms ``Linguistics" and ``Syntax", was the name chosen for a system that combines both the meta-language processor and
the user interface that makes use of such processor.
These case studies are also relevant as an example for a possible pedagogical scenario by using the system in a classroom context.
The purpose of this chapter is to also demonstrate some of the process in which both the Teacher and Student will be involved, and to exhibit some common structures
and inputs as well as their results.

\section{Attribute Validation}
This case study intends to demonstrate the validation on sentence components based on their attributes. The example showed in the previous chapter 
[\ref{lst:meta_struct_err}] contains a structure that is composed by two main parts: a subject (\underline{Sujeito}) and a predicate (\underline{Predicado}). 
The subject is then subdivided into a possible pronoun (\underline{Pronome}) and a noun (\underline{Nome}), which are then matched with a word 
(the lexical part identified by the student). The predicate is composed by a verb and a complement that is directly related to the verb. This complement 
(\underline{Complemento\_Direto}) is then composed by a possible determinant (\underline{Determinante}) and a mandatory noun (\underline{Nome}).

In this particular example, both the subject and the verb from the predicate have an attribute named ‘tipo’ which purpose is to check evaluate if each of the components
are animated or inanimated. By analysing the logic in the ERRORS block [\ref{lst:meta_struct_err}], we can see that if the value of the attribute ‘tipo’ is different 
between the two components, than an error should be pointed. In this case, the sentence parsed by the student is

``O Carlos teme a sinceridade.''

\noindent which is in fact a valid sentence, as the name ``Carlos'' and the verb ``teme'' are both animated.

When running the example above [\ref{lst:meta_struct_err} plus \ref{lst:meta_input}] in the Meta Grammar processor, and if no errors occur, a specific grammar (in ANTLR) 
in then generated. For this specific case, this is the generated grammar.

\begin{center}
\begin{minipage}{15cm}
\begin{lstlisting}[language=java, basicstyle=\tiny, label={lst:case_study_sentence}, caption=Example of a specific generated grammar.]
grammar Grammar;

@members {
    final String Sujeito__TIPO = ``animado";
    final String Predicado__Verbo__TIPO = ``animado";
}


main : sujeito predicado
{
    if ( Sujeito__TIPO.equals(``animado") && Predicado__Verbo__TIPO.equals(``inanimado") ) 
        { System.out.println(``ERROR!"); }

    if ( Sujeito__TIPO.equals(``inanimado") && Predicado__Verbo__TIPO.equals(``animado") ) 
        { System.out.println(``ERROR!"); }
}
;

sujeito : (determinante)? nome 
;

determinante : 'O' | 'a'
;

nome : 'Carlos' | 'sinceridade'
;

predicado : verbo complemento_direto 
;

complemento_direto : (determinante)? nome 
;

verbo : 'teme'
;


/* LEXER */
(...)
\end{lstlisting}
\end{minipage}
\end{center}

In the ‘main’ production, we can see the logical conditions that are ready to be evaluated when running this grammar. As the conditions are false, no errors should occur,
allowing for the visualization of the syntax tree.

% INSERIR PRINT DO TESTRIG????

%To conclude, if we use as an input a sentence like
%
%``O Carlos teme a sinceridade.''
%
%\noindent the sentence is accepted because, firstly, the structure is correct, and every component is in the right place, and secondly, all the attributes obey to the rules. As another example, if we use as an input other sentence for example
%
%``O acidente teme a sinceridade.''
%
%\noindent we get an error which indicates that the verb ``teme'' requires an animated subject, and ``acidente'' belongs to the subject that has an inanimated property.
%
%\begin{center}
%\begin{minipage}{10cm}
%\begin{lstlisting}[language=java, basicstyle=\small, label={lst:error_message_dsl}, caption=Example of an error message]
%ERRO:
%- O verbo 'teme' requer um sujeito ANIMADO.
%\end{lstlisting}
%\end{minipage}
%\end{center}

\section{Missing components \& Warnings}
Still based on the previous defined structure [\ref{lst:meta_struct_err}], the student's specification can still be missing some components that are mandatory. This case 
study just intends to show the way that errors or warnings are notified to the user.

In order to demonstrate this, the input block defined above [\ref{lst:meta_input}] is going to suffer some modifications in order to trigger errors or warnings.

\begin{center}
\begin{minipage}{13cm}
\begin{lstlisting}[language=java, basicstyle=\small, label={lst:meta_input_missing_comp}, caption=Example of the students parsing with missing component]
INPUT:
    - (Sujeito: ``O Carlos" [tipo = ``animado"]
        (Determinante: ``O"))
    - (Predicado: ``teme a sinceridade" 
        (Verbo: ``teme" [tipo = ``animado"], 
         Complemento_Direto: ``a sinceridade" 
            (Determinante: ``a", Nome: ``sinceridade")))
\end{lstlisting}
\end{minipage}
\end{center}

In this example, we can see that within the subject component (\underline{Sujeito}), the noun component (\underline{Nome}) is not defined. This particular case would
cause an error as the noun component is mandatory (based on the previous structure [\ref{lst:meta_struct_err}]). The error message identifies the missing component.

\begin{center}
\begin{minipage}{13cm}
\begin{lstlisting}[language=java, basicstyle=\small, label={lst:err_msg_missing_comp}, caption=Example error message of missing component]
ERROR: (INPUT) 
- The mandatory component 'Nome' has not been defined.
\end{lstlisting}
\end{minipage}
\end{center}
Another possible error would be to not define attributes, and to not give those attributes values. In this case, if we remove the attribute ‘tipo’ from the subject 
component (\underline{Sujeito}),

\begin{center}
\begin{minipage}{13cm}
\begin{lstlisting}[language=java, basicstyle=\small, label={lst:meta_input_missing_attr}, caption=Example of the students parsing with missing attribute]
INPUT:
    - (Sujeito: ``O Carlos"
        (Determinante: ``O", Nome: ``Carlos"))
    - (Predicado: ``teme a sinceridade" 
        (Verbo: ``teme" [tipo = ``animado"], 
         Complemento_Direto: ``a sinceridade" 
            (Determinante: ``a", Nome: ``sinceridade")))
\end{lstlisting}
\end{minipage}
\end{center}

\noindent the error should notify the user that the subject component is missing attributes, as attributes are always mandatory (if the component related to them is also
mandatory).

\begin{center}
\begin{minipage}{14cm}
\begin{lstlisting}[language=java, basicstyle=\small, label={lst:err_msg_missing_attr}, caption=Example error message of missing attributes]
ERROR: (INPUT)
- There are attributes related to the component 'Sujeito' that were not defined.
\end{lstlisting}
\end{minipage}
\end{center}

When it comes to warnings, there is only one case that raises them. This happens when the user defines the same attribute multiple times, warning that only the last value 
will be considered for the final evaluation. If, for example, we use the same attribute twice on the subject component,

\begin{center}
\begin{minipage}{14cm}
\begin{lstlisting}[language=java, basicstyle=\small, label={lst:meta_input_attr_twice}, caption=Example of the students parsing with the same attribute in a single component]
INPUT:
    - (Sujeito: ``O Carlos" 
        [tipo = ``animado", tipo = ``inanimado"]
        (Determinante: ``O", Nome: ``Carlos"))
    - (Predicado: ``teme a sinceridade" 
        (Verbo: ``teme" [tipo = ``animado"], 
         Complemento_Direto: ``a sinceridade" 
            (Determinante: ``a", Nome: ``sinceridade")))
\end{lstlisting}
\end{minipage}
\end{center}

\noindent a warning is raised to notify the user that only the last value was considered as final (\emph{tipo = ``inanimado''}).

\begin{center}
\begin{minipage}{14cm}
\begin{lstlisting}[language=java, basicstyle=\small, label={lst:meta_input_missing_attr_warn}, caption=Example warning message of same attribute in a single component]
WARNING: (INPUT) 
- The attribute 'tipo' has already been declared! Using the last value found.
\end{lstlisting}
\end{minipage}
\end{center}

\section{Arbitrary Structure}

This last case study has the intention to demonstrate that is possible to define any arbitrary sentence structure, without obeying to any specific linguistic rules. If, for instance, the main goal of the teacher is to test
different attributes despite of the components of a sentence, a simple structure can be defined for that same purpose. The following structure and rules intend to test the gender between two components, and this can be done 
with very simple sentences.

\begin{center}
\begin{minipage}{15cm}
\begin{lstlisting}[language=java, basicstyle=\tiny, label={lst:arbitrary_structure}, caption=Example of an arbitrary sentence structure]
STRUCTURE:
    part(
        Frase,
        subparts[
            (Determinante, attributes{genero}),
            (Nome, attributes{genero}),
            (Verbo)
        ]
    )

ERRORS:
    Frase.Determinante->genero = ``masculino" AND Frase.Nome->genero = ``feminino";
    Frase.Determinante->genero = ``feminino" AND Frase.Nome->genero = ``masculino";
    
    Frase.Determinante->genero != ``masculino" AND Frase.Determinante->genero != ``feminino";
    Frase.Nome->genero != ``masculino" AND Frase.Nome->genero != ``feminino";
\end{lstlisting}
\end{minipage}
\end{center}

Based on the rules written, we can see that the gender must be equal, or the sentence is invalid. Furthermore, the rules ensure that the gender can only be male or female (``masculino'' and ``feminino'' respectively) in order
to be a valid sentence.

\begin{center}
\begin{minipage}{14cm}
\begin{lstlisting}[language=java, basicstyle=\tiny, label={lst:arbitrary_structure_input}, caption=Example of an arbitrary sentence input]
INPUT:
    - (Frase: ``A Olinda come"
        (Determinante: ``A" [genero = ``feminino"],
         Nome: ``Olinda" [genero = ``feminino"],
         Verbo: ``come"))
\end{lstlisting}
\end{minipage}
\end{center}

Combining all the information in the processor, we generate a specific grammar for this arbitrary structure.

\begin{center}
\begin{minipage}{15cm}
\begin{lstlisting}[language=java, basicstyle=\tiny, label={lst:case_study_sentence}, caption=Example of a specific generated grammar.]
grammar Grammar;

@members {
    final String Frase__Determinante__GENERO = ``feminino";
    final String Frase__Nome__GENERO = ``feminino";
}


main : frase
{
    if ( Frase__Determinante__GENERO.equals(``masculino") && Frase__Nome__GENERO.equals(``feminino") ) 
        { System.out.println(``ERROR!"); }
	
    if ( Frase__Determinante__GENERO.equals(``feminino") && Frase__Nome__GENERO.equals(``masculino") ) 
        { System.out.println(``ERROR!"); }
	
    if ( !Frase__Determinante__GENERO.equals(``masculino") && !Frase__Determinante__GENERO.equals(``feminino") ) 
        { System.out.println(``ERROR!"); }
	
    if ( !Frase__Nome__GENERO.equals(``masculino") && !Frase__Nome__GENERO.equals(``feminino") ) 
        { System.out.println(``ERROR!"); }
}
;

frase : determinante nome verbo 
;

determinante : 'A'
;

nome : 'Olinda'
;

verbo : 'come'
;


/* LEXER */
(...)
\end{lstlisting}
\end{minipage}
\end{center}


This simple example shows that the meta-language created is flexible to the point of writing arbitrary sentences or rules, 
augmenting the possibilities of syntactic structures.

\section{Further examples and structures}

In order to demonstrate even further the capabilities of the tool, some more examples of simple (but concrete) and complex sentences will be included.
Some of the previous examples had the intent of demonstrating validations and/or trigger errors and warnings.
In these next examples, it is intended to explore more common tested structures and to touch on the topic of Complex Sentences,
showing that it is certainly possible to have them translated into the meta-language.


This first example is, again, composed by a Subject - Predicate structure, with some caracteristics to the Predicate itself.

\begin{center}
\begin{minipage}{14cm}
\begin{lstlisting}[language=java, basicstyle=\small, label={lst:example_structure1}, caption=Example of a sentence structure]
STRUCTURE:
    part[(Sujeito)]
    part[(
        Predicado,
        subparts[
            (Verbo),
            (Complemento_Direto),
            (Predicativo_Complemento_Direto)?
        ]
    )]
\end{lstlisting}
\end{minipage}
\end{center}

\begin{center}
\begin{minipage}{14cm}
\begin{lstlisting}[language=java, basicstyle=\small, label={lst:example_input1}, caption=Example of a sentence input]
INPUT:
    - (Sujeito: ``O rapaz")

    - (Predicado (
        Verbo: ``viu",
        Complemento_Direto: ``o homem",
        Predicativo_Complemento_Direto: ``com o telescopio"
    ))
\end{lstlisting}
\end{minipage}
\end{center}


The next example uses an attribute to limit the domain of the a component (\textit{Complemento\_Circunstancial}).

\begin{center}
\begin{minipage}{15cm}
\begin{lstlisting}[language=java, basicstyle=\small, label={lst:example_structure3}, caption=Example of a sentence structure]
STRUCTURE:
    part[(
        Frase, subparts[
            (Modificador),
            (Sujeito),
            (Predicado, subparts[
                (Verbo),
                (Complemento_Direto),
                (Complemento_Circunstancial, attributes{tipo})
            ])
        ]
    )]

ERRORS:
    Frase.Predicado.Complemento_Circunstancial->tipo != ``lugar";
\end{lstlisting}
\end{minipage}
\end{center}

\begin{center}
\begin{minipage}{15cm}
\begin{lstlisting}[language=java, basicstyle=\small, label={lst:example_input3}, caption=Example of a sentence input]
INPUT:
    - (Frase (
        Modificador: ``Hoje",
        Sujeito: ``eu",
        Predicado (
            Verbo: ``comi",
            Complemento_Direto: ``uma pizza",
            Complemento_Circunstancial: "na pizzaria abaixo" [tipo = ``lugar"]
        )
    ))
\end{lstlisting}
\end{minipage}
\end{center}


We can also use an ``\emph{\(|\)}" (\textit{or}) operator within the structure rules, giving the student the possibility of defining one or other component,
but maintaing the main structure of the sentence. 
Using an operator with this capability, we prevent the need of creating two separate structures only for a single change. 

\begin{center}
\begin{minipage}{15cm}
\begin{lstlisting}[language=java, basicstyle=\small, label={lst:example_structure4}, caption=Example of a sentence structure]
STRUCTURE:
    part[(
        Frase, subparts[
            (Interjeccion | Pronombre),
            (Verbo),
            (Nombre)
        ]
    )]
\end{lstlisting}
\end{minipage}
\end{center}

\begin{center}
\begin{minipage}{15cm}
\begin{lstlisting}[language=java, basicstyle=\small, label={lst:example_input4}, caption=Example of a sentence input]
INPUT:
    - (Frase (
        Interjeccion: ``Hola!",
        Verbo: ``Soy",
        Nombre: ``Manuel"
    ))
\end{lstlisting}
\end{minipage}
\end{center}


Complex Sentence...

\begin{center}
\begin{minipage}{14cm}
\begin{lstlisting}[language=java, basicstyle=\small, label={lst:example_complex_structure}, caption=Example of a complex sentence structure]
STRUCTURE:
    part[(Sujeito)]
    part[(
        Predicado,
        subparts[
            (Verbo),
            (Complemento_Indireto)
        ])
    )]
    part[(Conjuncao)]
	part[(
        Predicado,
        subparts[
            (Verbo),
            (Modificador_Verbal)
        ]
    )]
\end{lstlisting}
\end{minipage}
\end{center}


\begin{center}
\begin{minipage}{14cm}
\begin{lstlisting}[language=java, basicstyle=\small, label={lst:example_complex_input}, caption=Example of a complex sentence input]
INPUT:
    - (Sujeito: ``Los soldados")

    - (Predicado (
        Verbo: ``dispararam",
        Complemento_Indireto: ``a los sentenciados"
    ))

    - (Conjuncao: ``y")

    - (Predicado (
        Verbo: ``cayeron",
        Modificador_Verbal: ``muertos"
    ))
\end{lstlisting}
\end{minipage}
\end{center}
