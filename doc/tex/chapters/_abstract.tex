\begin{titlepage}
\renewcommand{\abstractname}{\vspace{-\baselineskip}}

% english
\begin{abstract}
\section*{Abstract}

This document presents the topic “Applying Attribute Grammars to teach Linguistic Rules”, at Universidade do Minho in Braga, Portugal.
This thesis is focused on using the formalisms of attribute grammars in order to create a tool to help linguistic students learn the different rules of a natural language. 
The system developed, named \textbf{Lyntax}, consists in a processor for a domain specific language which intends to enable the user to specify different kinds of
sentence structures, and afterwards, test various phrases against said structures.
The processor validates and evaluates the input given, generating a grammar which is specific to a previously chosen sentence.
Lastly, using \textsc{ANTLR}, a parser is generated for this grammar, and finally a syntax tree is presented to the user for analysis purposes.

An interface that supports the specification of the language was built, also allowing the use of the processor and the generation a grammar,
abstracting the user of any calculations.

Within this document, the focus will be primarly dedicated to the analysis of the system and how each block was built.
Different examples of the processor in action will be shown and explained.\\[0.1cm] 

%The main goal is to create a new DSL with a simple notation,
%suitable for any person that does not have any experience with common programming languages elements, that will allow to define linguistic exercices. 
%Furthermore, it was created a user interface that supports that same tool, granting a more visual experience for the user.\\[0.1cm]

\noindent \textbf{Keywords}: Linguistic, Natural Language Processing, Attribute Grammars

\end{abstract}


%%%%%%%%%%%%%%%%%%%%%%%%%%%%%%%%%%%%%%%%%%%%%%%%%%%%%%%%%%%%%%%%%%%%%%%%%%%%%%%%%%%%%%%%%%%%%%%%%%%%%%%%


% portuguese
\begin{abstract}
\section*{Resumo}

Este documento refere-se a uma dissertação sobre o tópico ``Aplicar Gramáticas de Atributos no ensino de Regras de Linguística'', 
e será concluída na Universidade do Minho em Braga, Portugal.
Esta dissertação pretende focar-se no uso dos formalismos das gramáticas de atributos de maneira a criar uma ferramenta que ajude os alunos de linguística a aprender 
as diversas regras da língua natural.

O sistema desenvolvido, denominado de \textbf{Lyntax}, consiste em um processor para uma linguagem de domínio específico cujo objetivo é o de permitir ao seu utilizador
a possibilidade de especificar diversas estruturas de frases, e posteriormente, testar frases contra essas mesmas estruturas.
O processador valida e avalia o input recebido, gerando uma gramática específica à frase previamente escolhida.
Por fim, usando uma ferramenta como o \textsc{ANTLR}, um parser é gerado para a gramática específica, e finalmente a árvore de syntax é apresentada ao utilizador com o
intuito de ser analisada.

For também criada uma interface que suporta a específicação da linguagem, permitindo também o uso do processador e a geração da gramática específica,
abstraindo assim o utilizador de quaiquer cálculos.

Neste documento, o focus primário será dedicado à análise do sistema e como cada bloco foi construído.
Diferentes exemplos de uso do processador serão apresentados e explicados.\\[0.1cm]

%O principal objetivo é a criação de uma \textsc{DSL} com uma notação simples, 
%adequada a qualquer pessoa que não tenha experiência com elementos comuns das linguagens de programação,
%que permita a definição de exercícios de linguística. 
%Para além disso, é esperada a criação de uma interface que sirva de suporte a essa mesma ferramenta, permitindo assim uma experiência visual ao utilizador.\\[0.1cm]
    
\noindent \textbf{Palavras-chave}: Linguística, Processamento de Língua Natural, Gramáticas de Atributos

\end{abstract}

\end{titlepage}
