\begin{titlepage}
\renewcommand{\abstractname}{\vspace{-\baselineskip}}

% english
\begin{abstract}
\section*{Abstract}

This document presents a proposal for a Master Thesis within the topic of “Applying Attribute Grammars to teach Linguistic Rules”, and will be accomplished at Universidade do Minho in Braga, Portugal.

\noindent This thesis is focused on using the formalisms of attribute grammars in order to create a tool to help linguistic students learn the different rules of a natural language. The main goal is to create a new \textsc{DSL} with a simple notation, suitable for any person that does not have any experience with common programming languages elements. Furthermore, it is expected to create a user interface that supports that same tool, granting a more visual experience for the user.\\[0.1cm]
    
\noindent \textbf{Keywords}: Linguistic, Natural Language, Attribute Grammars

\end{abstract}

%%%%%%%%%%%%%%%%%%%%%%%%%%%%%%%%%%%%%%%%%%%%%%%%%%%%%%%%%%%%%%%%%%%%%%%%%%%%%%%%%%%%%%%%%%%%%%%%%%%%%%%%

% portuguese
\begin{abstract}
\section*{Resumo}

Este documento refere-se a uma dissertação sobre o tópico ``Aplicar Gramáticas de Atributos no ensino de Regras de Linguística'', e será concluída na Universidade do Minho em Braga, Portugal.
	
\noindent Esta dissertação pretende focar-se no uso dos formalismos das gramáticas de atributos de maneira a criar uma ferramenta que ajude os alunos de linguística a aprender as diversas regras da língua natural. O principal objetivo é a criação de uma \textsc{DSL} com uma notação simples, adequada a qualquer pessoa que não tenha experiência com elementos comuns das linguagens de programação. Para além disso, é esperada a criação de uma interface que sirva de suporte a essa mesma ferramenta, permitindo assim uma experiência visual ao utilizador.\\[0.1cm]
    
\noindent \textbf{Palavras-chave}: Linguística, Língua Natural, Gramáticas de Atributos

\end{abstract}

\end{titlepage}